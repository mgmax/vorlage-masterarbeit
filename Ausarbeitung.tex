\documentclass[draft]{max-masterarbeit} % draft entfernen, um die "Entwurf"-Markierung in der Fußzeile zu deaktivieren

\subject{Masterarbeit}
\title{Toller Titel}
\author{Maximilian Gaukler}

\newcommand{\lehrstuhl}{Lehrstuhl für Regelungstechnik\\Universität Erlangen-Nürnberg}%
\newcommand{\datumTitelseite}{April 1234}
\date{02.03.1234}

\hyphenation{Rausch-an-teil Kor-re-lation Ab-schät-zun-gen Zu-stands-ab-hängig-keit}

%\includeonly{latenzen}


\begin{document}
\frontmatter
\pagenumbering{alph} % Seitennummerierung anfangs alphabetisch, damit nicht das Deckblatt die gleiche Seitennummer hat wie eine spätere Inhaltsseite

%\cleardoublepage
\titelseiteLRT
\pagestyle{empty}
\cleardoublepage
\chapter*{Aufgabenstellung}
\thispagestyle{empty}
\todo{Aufgabenstellung}
\cleardoublepage
\chapter*{Erklärung}
\thispagestyle{empty}
Ich versichere, dass ich die vorliegende Arbeit ohne fremde Hilfe und ohne Benutzung anderer als
der angegebenen Quellen angefertigt habe und dass die Arbeit in gleicher oder ähnlicher Form noch
keiner anderen Prüfungsbehörde vorgelegen hat und von dieser als Teil einer Prüfungsleistung
angenommen wurde. Alle Ausführungen, die wörtlich oder sinngemäß übernommen wurden, sind als 
solche gekennzeichnet.

\vspace{4.5em}

\makeatletter
\let\thedate\@date
\let\theauthor\@author
\makeatother
\newcommand{\unterschriftsbreite}{.35\linewidth}
%
\begin{tikzpicture}[every node/.style={inner sep=0, outer sep=0}]
	\node[anchor=south west] (o) at (0,0) {Erlangen, den \thedate};
	\coordinate (rechtsmitte) at ($(\linewidth - 1, 0)$);
	\draw (o)  (rechtsmitte) -- ++(-\unterschriftsbreite, 0);
	\node[text width=\unterschriftsbreite, align=center, anchor=north east, yshift=-.5em] (author) at (rechtsmitte) {\theauthor};
\end{tikzpicture}

\cleardoublepage
\pagenumbering{roman}
\pagestyle{scrheadings}
\tableofcontents



\mainmatter
\chapter{Einleitung}

\todo{... einleitende Worte...}

\chapter{asdf}
\blindtext \cite{beispielquelle}

\blindtext[42]

\chapter{Zusammenfassung}
\todo{}

\appendix
\cleardoublepage
%\chapter{Literatur}
\bibliographystyle{alphadin}

\addcontentsline{toc}{chapter}{Literaturverzeichnis}

\bibliography{literatur} % -> literatur.bib

\chapter{Verzeichnis der Formelzeichen}

\begin{tabular}{ll}
$d$ & Störeingang \\
$m$ & Dimension des Eingangs \\
$\Delta t$ & Verzögerungszeit \todo{dummy}
\end{tabular}

\chapter{Anhang}

\todo{}

\end{document}