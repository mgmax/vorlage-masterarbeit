\documentclass[draft]{max-masterarbeit} % draft entfernen, um die "Entwurf"-Markierung in der Fußzeile zu deaktivieren
\usepackage{textcomp} % für listings benötigt
\usepackage[final]{listings} % in draft mode wird nix angezeigt
% Workaround gegen Leerzeichen im Kopieren-Einfügen von Listings im PDF
\lstset{columns=fullflexible}
% literate: Workaround für listings + UTF8
% von https://en.wikibooks.org/wiki/LaTeX/Source_Code_Listings
% mit Ergänzungen
\lstset{literate=
  {á}{{\'a}}1 {é}{{\'e}}1 {í}{{\'i}}1 {ó}{{\'o}}1 {ú}{{\'u}}1
  {Á}{{\'A}}1 {É}{{\'E}}1 {Í}{{\'I}}1 {Ó}{{\'O}}1 {Ú}{{\'U}}1
  {à}{{\`a}}1 {è}{{\`e}}1 {ì}{{\`i}}1 {ò}{{\`o}}1 {ù}{{\`u}}1
  {À}{{\`A}}1 {È}{{\'E}}1 {Ì}{{\`I}}1 {Ò}{{\`O}}1 {Ù}{{\`U}}1
  {ä}{{\"a}}1 {ë}{{\"e}}1 {ï}{{\"i}}1 {ö}{{\"o}}1 {ü}{{\"u}}1
  {Ä}{{\"A}}1 {Ë}{{\"E}}1 {Ï}{{\"I}}1 {Ö}{{\"O}}1 {Ü}{{\"U}}1
  {â}{{\^a}}1 {ê}{{\^e}}1 {î}{{\^i}}1 {ô}{{\^o}}1 {û}{{\^u}}1
  {Â}{{\^A}}1 {Ê}{{\^E}}1 {Î}{{\^I}}1 {Ô}{{\^O}}1 {Û}{{\^U}}1
  {œ}{{\oe}}1 {Œ}{{\OE}}1 {æ}{{\ae}}1 {Æ}{{\AE}}1 {ß}{{\ss}}1
  {ű}{{\H{u}}}1 {Ű}{{\H{U}}}1 {ő}{{\H{o}}}1 {Ő}{{\H{O}}}1
  {ç}{{\c c}}1 {Ç}{{\c C}}1 {ø}{{\o}}1 {å}{{\r a}}1 {Å}{{\r A}}1
  {€}{{\EUR}}1 {£}{{\pounds}}1
  {*}{{\char42}}1 {-}{{\char45}}1
}
\lstset{upquote=true}

\subject{Masterarbeit}
\title{Toller Titel}
\author{Arnold Autor}

\newcommand{\lehrstuhl}{Lehrstuhl für Regelungstechnik\\Universität Erlangen-Nürnberg}%
\newcommand{\datumTitelseite}{April 1234}
\date{02.03.1234}

\hyphenation{Rausch-an-teil Kor-re-lation Ab-schät-zun-gen Zu-stands-ab-hängig-keit}

%\includeonly{latenzen}


\begin{document}
\frontmatter
\pagenumbering{alph} % Seitennummerierung anfangs alphabetisch, damit nicht das Deckblatt die gleiche Seitennummer hat wie eine spätere Inhaltsseite

%\cleardoublepage
\titelseiteLRT
\pagestyle{empty}
\cleardoublepage
\chapter*{Aufgabenstellung}
\thispagestyle{empty}
\todo{Aufgabenstellung}
\cleardoublepage
\chapter*{Erklärung}
\thispagestyle{empty}
Ich versichere, dass ich die vorliegende Arbeit ohne fremde Hilfe und ohne Benutzung anderer als
der angegebenen Quellen angefertigt habe und dass die Arbeit in gleicher oder ähnlicher Form noch
keiner anderen Prüfungsbehörde vorgelegen hat und von dieser als Teil einer Prüfungsleistung
angenommen wurde. Alle Ausführungen, die wörtlich oder sinngemäß übernommen wurden, sind als 
solche gekennzeichnet.

\vspace{4.5em}

\makeatletter
\let\thedate\@date
\let\theauthor\@author
\makeatother
\newcommand{\unterschriftsbreite}{.35\linewidth}
%
\begin{tikzpicture}[every node/.style={inner sep=0, outer sep=0}]
	\node[anchor=south west] (o) at (0,0) {Erlangen, den \thedate};
	\coordinate (rechtsmitte) at ($(\linewidth - 1, 0)$);
	\draw (o)  (rechtsmitte) -- ++(-\unterschriftsbreite, 0);
	\node[text width=\unterschriftsbreite, align=center, anchor=north east, yshift=-.5em] (author) at (rechtsmitte) {\theauthor};
\end{tikzpicture}

\cleardoublepage
\pagenumbering{roman}
\pagestyle{scrheadings}
\tableofcontents



\mainmatter
\chapter{Einleitung}

\todo{... einleitende Worte...}

ölölölölölölölö Ümläute



\chapter{TODO}
\todo{Seitenränder sind nicht ideal...}

\todo{Schriftarten gibt es auch bessere...}

\todo{BUG: Titelzeile im header ist falsch (arabische Nummern statt Buchstaben vor Anhangskapiteln)}

\todo{Bindekorrektur nochmal genau nachprobieren, ist jetzt eher minimal zu groß (bei Leimbindung). Bei Ringbindung müsste man sie ganz neu austüfteln.}

\chapter{Mathematik und Schriftartentest}
Bei Nicht-Standard-Schriftart sehen insbesondere die \emph{mathcal} Zeichen wie $\mathcal{E}\left\{x+y\right\}$ für Erwartungswert oder $\dot x = \mathcal{A} x + \mathcal{B} u$ für verteiltparametrische Systeme manchmal komisch oder zu verschnörkelt aus.

Die Zeit $\tau$ demonstriert die Bitterkeit der Welt, wenn das Zeichen $\tau$ sich nicht sauber von der Schriftdicke in den Text einpasst. Der Term $f(x)=\sin x$ wird auch als \emph{Weltformel} bezeichnet, wenn einem gerade kein größerer Unsinn einfällt.



\chapter{asdf}
\blindtext \cite{beispielquelle}

\blinddocument

\chapter{Zusammenfassung}
\todo{}

\appendix
\cleardoublepage
%\chapter{Literatur}
\bibliographystyle{alphadin}

\addcontentsline{toc}{chapter}{Literaturverzeichnis}

\bibliography{literatur} % -> literatur.bib

\chapter{Verzeichnis der Formelzeichen}

\begin{tabular}{ll}
$d$ & Störeingang \\
$m$ & Dimension des Eingangs \\
$\Delta t$ & Verzögerungszeit \todo{dummy}
\end{tabular}

\chapter{Matlab Code}

\todo{}
%\blindtext[8]
ööö
\lstset{language=Matlab}

\begin{lstlisting}[frame=single]
% Das Wetter ist schön.
temperatur = 35;

% Regelungstechnik ist noch schöner.
temperatur_soll = 25;
a = [
    0 1 0
    0 -1 1
    7 0 -1
    ];
c = [ 0 1 0 ];

b = [ 1 0 1 ]';

c*(a-sym('s')*eye(3))^-1*b
rank(obsv(a,c))
rank(ctrb(a,b))
\end{lstlisting}

Ziel der Vorlage: Das Listing ist aus Acrobat PDF Reader durch Kopieren und Einfügen 1:1 übernehmbar.
Zur Zeit noch nicht ganz erreicht.

\lstlistoflistings

fsad8i

\end{document}
