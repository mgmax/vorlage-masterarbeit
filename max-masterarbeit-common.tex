% Kodierung und Ausgabeformat:
\usepackage[ngerman]{babel} % dt. Zeichentrennung
\usepackage{lmodern} % TODO Schriftart
%\usepackage{newtxtext,newtxmath}
%\usepackage{charter}
%\usepackage[expert]{mathdesign} % TODO Schriftart
\usepackage[T1]{fontenc} % interne Schriftkodierung von Tex
\usepackage[utf8]{inputenc} % Zeichenkodierung der Dateien UTF8
\usepackage[draft=false]{hyperref} % anklickbare Links (Seitenzahlen etc.) im PDF
\usepackage[final]{graphicx}

\usepackage[usenames,dvipsnames,svgnames]{xcolor} % allerlei Features in \color, z.B. Mischen von Farben mit \color{blue!30!black}
\color{black} % damit nicht Text manchmal als dunkelstgrau rauskommt

% Entwurfsmodus - Warnung
% der draft mode markiert überlange Zeilen mit schwarzen Kästchen am rechten Rand
% Zu deaktivieren für die fertige Fassung, d.h das [draft] aus \documentclass... im Hauptdokument entfernen
\usepackage{ifdraft}
% Bitte als \documentclass[draft]{...} für Entwürfe laden und das draft weglassen, sobald die fertige Datei erstellt wird
\ifdraft{
\ClassWarningNoLine{}{Entwurfsmodus [draft] ist an. Bitte ausschalten, sobald das Dokument fertig ist.}
}{}






% Feature-Pakete
\usepackage{csquotes} % für \enquote{Text in Anführungszeichen}
% für \text in Matheumgebungen.
% fleqn = \equation ist linksbündig
\usepackage[fleqn]{amsmath}
% \mathbb{R} und weitere Symbole
\usepackage{amssymb}
\usepackage{mathtools}
\usepackage{placeins} % für \FloatBarrier
\usepackage{floatflt} % für floatingfigure
\usepackage{subfig} % \subfloat für Unter-figures, \subref für Bezüge darauf

\usepackage[german]{cleveref}

\usepackage{tikz}
\usetikzlibrary{positioning}
\usetikzlibrary{calc}
\tikzstyle{pfeil}=[->, semithick]
\tikzstyle{pfeilR}=[pfeil, <-]
\tikzstyle{block}=[rectangle,draw,thick, minimum height=2em]


\usepackage{blindtext}
\usepackage{siunitx} % korrektes Setzen von Einheiten mit \SI{1234}{km} und auch Komma-Zahlen wie \SI{1.4e12} -> 1,4 . 10^12
\sisetup{locale=DE}

\ifdraft{
\newcommand{\todo}[1]{{\color{black!60} [TODO #1]}}
\newcommand{\randnotiz}[1]{\marginpar{\raggedleft #1}}
}{
\ClassWarningNoLine{}{Entwurfsmodus [draft] ist aus, TODO und randnotiz Kommandos sind deaktiviert (TODO: stattdessen nur konkrete Warnungen).}
}



% TODO \titlehead{{\Large Universit"at Schlauenheim \hfill SS~2001\\}
%Institut f"ur Raumkr"ummung\\
%Hochschulstra"se~12\\
%34567 Schlauenheim}

% Eulersche Zahl
\newcommand{\ee}{\mathrm{e}}

% 2-zeiliger Vektor/Matrix
% [ #1
%   #2 ]
\newcommand{\vektorZwei}[2]{\left[ \begin{matrix} #1 \\ #2 \end{matrix} \right]}
% beliebige Matrix in eckigen Klammern
\newcommand{\vektor}[1]{\left[ \begin{matrix} #1 \end{matrix} \right]}


% Erwartungswert
\newcommand{\ew}[1]{\mathop{\mathcal{E}}\!\left\{#1\right\}}
% Erwartungswert bzgl Variable \ewBezug{v}{Funktion(v)}
\newcommand{\ewBezug}[2]{\mathop{\mathcal{E}_{#1}}\!\left\{#2\right\}}

% R: Menge der reellen Zahlen
\newcommand{\R}{\mathbb{R}}
% gerades d für Integrale/Ableitungen:
% \diff{x} für dx + passenden Leerraum
\newcommand{\diff}[1]{\mathrm{d}#1\,}

\newcommand{\mathupright}[1]{\mathrm{#1}}
\newcommand{\spur}{\mathupright{spur}}
\newcommand{\diag}{\mathupright{diag}}

\newcommand{\titelseiteLRT}{
	\begin{titlepage}
	\makeatletter
	\begin{tikzpicture}[remember picture,overlay,every node/.style={text width=.9\linewidth, align=center}]
			\node[below = 5.9cm of current page.north] (title)  {\Large \bfseries \@title };
		\node[text width=10cm, align=center, below=6cm of title.north] (arbeitstyp) {Als \@subject\\vorgelegt von};
		\node[below=3.25cm of arbeitstyp.north] (author) {\large \bfseries \@author};
		\node[below=8cm of author.north] (lehrstuhl) {\lehrstuhl};
		\node[below=3.8cm of lehrstuhl.north] {\datumTitelseite};
	\end{tikzpicture}
	\makeatother
	\end{titlepage}
}
