
% Typographie für A4-Ausgabe (nicht Präsentationen) mit KOMAscript basierten Klassen (scrbook, scrartcl)
\usepackage[draft=false]{scrlayer-scrpage} % \cfoot etc. für Kopf/Fußzeilenelemente. Hier draft=false weil sonst "Lineale" für verschiedene Seitenebenen eingezeichnet werden.


\pagestyle{scrheadings}
\newcommand{\entwurfsmarkierung}{\ifdraft{\textsl{Entwurf, Stand \today}}}
\cfoot[\entwurfsmarkierung]{\entwurfsmarkierung}

\KOMAoptions{paper=a4, fontsize=12pt}
\KOMAoptions{headsepline, footsepline, plainfootsepline} % Querlinien als Trennung zwischen Inhal und Kopf- und Fußzeile; Fußzeilenlinie auch bei Kapitelanfang (Seitenstil plain)
\setkomafont{pageheadfoot}{\scshape} % Kopf- und Fußzeile in Kapitälchen
\KOMAoptions{BCOR=18mm} % Bindekorrektur: wieviel wird innenliegende Rand einer halben Doppelseite durch die Bindung und das "Knicken" einer Seite kleiner?
\usepackage[onehalfspacing]{setspace} % Zeilenhöhe 1.5 
\usepackage[final,activate={true,nocompatibility},shrink=10,stretch=10]{microtype} % Schrift minimalst stauchen/dehnen, damit Silbentrennung besser klappt



% Am Anfang des Kapitels die Überschrift als "Kapitel 2 \n Reglerentwurf" setzen
\KOMAoptions{headings=twolinechapter}
% in der Kopfzeile aber nur als "2 Reglerentwurf"
\renewcommand*{\chaptermarkformat}{\arabic{chapter} }

% Satzspiegel - hingebastelt damit es ähnlich zur alten LRT Vorlage ist.
\KOMAoptions{footinclude=false}
\KOMAoptions{DIV=14} % Satzspiegel - DIV ist ein ganzzahliger "Frickelfaktor", je größer desto weniger Seitenrand. Das Verhältnis der Seitenränder zueinander folgt dann typographischen Vorgaben, man sollte es deshalb nicht antasten.
% TODO der Wert 14 kommt der alten Vorlage am nächsten, typographisch wäre ggf. ein größerer automatisch berechneter Rand DIV=calc sinnvoll.
\KOMAoptions{titlepage=firstiscover} % TODO
\KOMAoptions{parskip=half-} % TODO Absatzabstand... ungefähr der alten Vorlage nachgemacht
\KOMAoptions{numbers=noenddot} % keine Punkte nach Nummerierungen wie Kapitelüberschrifen oder Abbildungen (z.B. 1 Einleitung statt 1. Einleitung)
